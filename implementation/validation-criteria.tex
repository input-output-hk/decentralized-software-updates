
\section{Validation criteria}
\label{sec:validation-criteria}

This section describes the tests performed on the update protocol implementation
to validate its correctness.

The update mechanism is modeled using data-automata and implemented as a set of
structured-operational semantic rules~\cite{7436dacbd7664737925923f0657fdc07,
  quickcheck-state-machine}. The conformance of the implementation w.r.t. its
specification is tested by means of property-based
tests~\cite{10.1145/1988042.1988046}.

In this section we introduce the data-automata formalism, show how conformance
of the implementation w.r.t. automata models is carried out, and present the
data-automata models that capture the properties described in
Section~\ref{sec:cand-properties}.

\subsection{The data automata formalism}
\label{sec:the-data-automata-formalism}

TODO.

\subsection{Conformance tests}
\label{sec:conformance-tests}

TODO.

\subsection{Data-automata models for the update protocol}
\label{sec:automata-models-update-protocol}


\begin{itemize}
\item We assume a correspondence between natural numbers and SIP's.
\item Models are quantified over SIP's.
\end{itemize}

TODO: show full model (quantified over the maximum number of SIP's).

\subsubsection{Ideation phase}
\label{sec:validation-ideation-phase}

\paragraph{Property~\ref{prop:unique-sip}} We have validated that the
implementation conforms to the commit-reveal model shown in Equation~??? and
Figure~???, which is part of the full model of Equation~???. The model in
Equation~??? can perform a $\mathsf{submit}_i$ action only once. Here it is
clear that, since this model synchronizes on the $\mathsf{submit}_i$ action,
which means that no other automata can submit the same SIP without synchronizing
with this automata, therefore in the full model a submission for an SIP can
occur at most once (remember that since we test conformance of the rules w.r.t.
invalid automata traces as well, this means that duplicated SIP's should also be
rejected by a system that conforms to the full model of Equation~???).

TODO: write the equation corresponding to the commit reveal automaton
synchronizing on submit and reveal action.

TODO: generate commit-reveal figure from Haskell implementation.

\paragraph{Property~\ref{prop:reveal-valid}} This property is also captured by
the commit-reveal automata discussed in the previous section (Equation~??? and
Figure~???).

\paragraph{Property~\ref{prop:active-sip-valid}} This is captured by the
\texttt{activation} model described in Equation~??? and Figure~??? As it can be
observed in this model, SIP's are only active once the $\mathsf{reveal}_i$
signal is stable, and time can pass up to the point in which the end of the
voting period ends. After that, the SIP is not active anymore.

TODO: generate the activation model and add an equation with the corresponding
synchronizing actions.

\paragraph{Property~\ref{prop:sip-votes-valid}} This requirement also is
validated by showing conformance to the \texttt{activation} model.

\paragraph{Property~\ref{prop:sip-tally-valid}} This requirement is validated by
the parallel composition and synchronization of the \texttt{activation} and
\texttt{voteTally} models. Here the $\mathsf{tally}_i$ action can only occur
after the end of the vote period is stable, and stake updates are not possible
while tallying, which means that the stake considered when tallying is the state
at the slot when the end of the voting period stabilized. Note that time could
pass while tallying (we do not restrict this), we only care about the snapshot
of the stake we use in tallying.

\paragraph{Property~\ref{prop:sip-tally-result}} This requirement is validated
by the \texttt{voteTally} model.

\paragraph{Property~\ref{prop:sip-revote}} TODO: we need to model revoting.


%%% Local Variables:
%%% mode: latex
%%% TeX-master: "decentralized-updates"
%%% End:
