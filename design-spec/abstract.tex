\begin{abstract}
%\subsection*{Abstract}
Software updates are a synonym to software evolution and thus are ubiquitous
and inevitable to any blockchain platform. In this document, we propose a general
framework for decentralized software updates in distributed ledger systems. Our
framework is primarily focused on Proof of Stake blockchains and aims at
providing a solid set of enhancements, covering the full spectrum of a
blockchain system, in order to ensure a decentralized, but also secure update
mechanism for a public ledger.
%
We identified what are the critical decisions in the lifecycle of a software
update
and then proposed a secure software update protocol that covers the \emph{full
lifecycle} of a software update from the ideation phase (the moment in which a
change to the blockchain protocol is proposed) to the actual activation of the
updated blockchain protocol, which enables decentralized decision making for
\emph{all} critical decisions.
%
We proposed a liquid democracy scheme based on
\emph{experts} for all the critical, but also deeply technical, decisions for
software updates. We formally defined what it means for a \emph{decentralized}
software update system to be secure and propose secure activation protocols
\cite{secure_activation} with various trade-offs. We proposed a design for
dealing with the complexity of priorities, version dependencies, conflicts
resolution and emergency handling for the activation of updates.
%
We performed voting and activation threshold analysis, in order to achieve both
properties of \emph{safety} and \emph{liveness}, which we define in detail.
%
We carried out several performance analyses that showed our update protocol
is both linearly scalable in the number of participants and does not impact
negatively the performance of the underlying blockchain.
%
We implemented our ideas into a prototype and
propose an architecture for integrating an update mechanism within the Cardano
node \cite{cardano}
%
The prototype was validated using a trace property-based testing
framework that we developed during this project.
%
To the best of our knowledge, this is
the first work that aims at formalizing the notion of a decentralized secure
update for a blockchain and also that takes such a holistic approach on
software updates.
\end{abstract}
