\section{Proof of Theorem~\ref{th:main}}\label{se:proof}
\begin{proof}
To simplify our proof we introduce the notion of \emph{canonical scenario} for the the ledger $\ledger_2$. In a canonical scenario the ledger
$\ledger_2$ is executed in the standard way. More precisely, 
we assume the existence of a genesis block and that $\asmp_2[\tau]$=1 for all $\tau \geq 0$. Let $\parties$ be the set of parties that is running $\ledger_2$. Also, 
let $t_j$ be the smallest time slot in which $B_j$ appears in  $\state_2^{P_i}[t_j]$ for each $P_i\in\parties$ and let $t_{j+k}$ be smallest time slot in which $B_{j+k}$ appears in %\nnote{$\state_2^{P_i}[t_{j+k}]$}
$\state_2^{P_i}[t_{j+k}]$ for each $P_i\in\parties$.
We are now ready to prove the security of $\Pi$.

In the protocol $\Pi$, by assumption, we have that $\asmp_2[\tau]=1$ for all $\tau \geq t$. 
%From the description of $\Pi$ we can claim that $t \leq t_{P_1}+k\cg^{-1}=t_{P_1}+\Delta_1$.
%That is, when an honest party is activated she waits to be sure that also the other honest parties are activated. %Hence, from the chain-growth and the common-prefix parameters
%each party needs to wait at most $\Delta_1=k\cg^{-1}$ time slots.
From the moment when $\asmp_2$ becomes true the activation process takes $\Delta\leq k \cg^{-1} + k \cg^{-1}$  time slots to be completed.
This is because the parties need to wait that the $(j+k)$-th block of $\state_1$ is part of $\check \state_1^{P_i}[t]$  for all $P_i$ and that $k$ blocks are generated in $\state_2$. Note that when $k$ blocks are generated in $\state_2$ at least 
$k$ blocks are generated in $\state_1$ since $\ledger_2$ and $\ledger_1$ have the same parameters and that the honest parties that maintain $\ledger_1$ are greater or equal than the parties that maintain $\ledger_2$. Therefore, the parties need to wait the \emph{special genesis block} of $\ledger_2$ to appear 
in $\state_2^P$ for each honest $P\in\activep$. Given that a block in $\ledger_1$ ($\ledger_2$) takes at most $\cg^{-1}$ time slots then we have that $\Delta \leq k \cg^{-1} + k \cg^{-1}$.
In the moment that a \emph{candidate} block $B^i_{j+k}$ becomes available to an honest party $P_i\in\activep$ (i.e., $B^i_{j+k}$ is part of $\check \state_1^{P_i}$) then she starts running $\ledger_2$
using ${B^i}'$ which is computed from $B^i_{j+k}$ as described earlier (we recall that at this time slot the assumption $\asmp_2$ holds).
Let $t'$ be the smallest time slot in which $B_{j+k}$ appears in $ \state_2^{P_i}[t']$
%\nnote{$\state_2^{P_i}[t']$}
 for each $P_i\in\parties$.
If we take the execution of the protocol from time $t$ and $t'$ this can be seen as a canonical execution of $\ledger_2$ given
that the parameters of $\ledger_1$ and $\ledger_2$ are the same. The only difference between this and the canonical scenario is
that the blocks $B_{j}, B_{j+1},\dots, B_{j+k}$  are generated using $\ledger_1$, but this does not represent an issue since we are assuming that
any block of $\state_1$ can be turned into a block of $\state_2$. \end{proof}









