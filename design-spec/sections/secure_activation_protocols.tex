\subsection{Secure Activation Protocols}\label{sec:secure_activation_protocols}
A secure activation protocol is one that achieves a secure transition from the 
current version of the consensus protocol to a new one. We have formally 
defined what is a \emph{secure activation} and have proposed two distinct 
protocols that provably achieve this. In a
nutshell, \emph{secure activation} means, the secure transition from the old
ledger (L1) to the new ledger (L2) in a way where:
\begin{itemize}
	\item L2 enjoys liveness \cite{backbone}
	\item L2 enjoys consistency \cite{backbone}
	\item L2 has L1 as a prefix
\end{itemize}

Our first activation protocol requires the structure of the current and the 
updated blockchain to be very similar (only the structure of the blocks can be 
different) but it allows for an update process more simple and efficient. The 
second activation protocol that we propose is very generic (i.e., makes few 
assumptions on the similarities between the structure of the current blockchain 
and the updated blockchain). The drawback of this protocol is that it requires 
the new blockchain to be resilient against a specific adversarial behaviour and
requires all the honest parties to be online during the update process.
However, we show how to get rid of the latest requirement (the honest
parties being online during the update) in the case of proof-of-work and
proof-of-stake ledgers. The interested reader can find more details on this 
topic in our paper \cite{secure_activation}.