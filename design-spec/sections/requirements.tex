\section{Requirements for a Decentralized Software Update 
System}\label{sec:requirements}

%\subsection{The Cardano software update system vision	
%statements}\label{sec:vision_statements}
\begin{itemize}
	\item[\textbf{Open Participation Enabled}] Anyone who can submit a common
	transaction should be able to submit an update proposal and vote for an
	update proposal.
	
	\item[\textbf{Decentralized Decision-Making Enabled}]  All major decisions
	typically made by central authorities (e.g., code maintainer, software
	owner etc.)
	should
	be made
	collectively by the community via a voting mechanism: a) What proposal will
	move forward? b) Do we accept the submitted implementation? and c)
	will the changes be activated?
	
	\item[\textbf{Protocol-Driven}] The update mechanism should be an on-chain
	protocol driven process: Not
	based on informal discussions (social consensus\footnote{Although this is
		also a very useful preparatory step to utilize along with the update
		protocol.}), but based on a protocol
	with specific security guarantees; a protocol that will leave behind an
	immutable (on-chain) trace of events
	
	\item[\textbf{Transparent and Auditable}] Anybody should be able to answer
	\emph{when}, \emph{why} and \emph{how} the system has evolved the way it
	has. The evolution history of the system should be open to everyone and
	form a globally-consistent tamper-evident public release log.
	
	\item[\textbf{Secure Activation Enabled}] The activation of changes on the 
	blockchain system should be resilient to chain-splits and ensure a secure 
	transition from the previous version of the consensus protocol to the new 
	one. We need to formally define what is a \emph{secure
		activation} and propose a protocol that enables such an activation.
	
	\item[\textbf{Performant and Scalable}] The update mechanism should have a 
	minimal impact on the transaction throughput and size of the underlying
	blockchain. Moreover, the update mechanism should be scalable to thousands
	(or even
	millions) of participants.
	
	\item[\textbf{Metadata-Driven Update Logic}] All updates are not the same
	(criticality,
	impact on the system, time to deploy etc.). We should enable a
	metadata-driven update policy (voting periods length, activation priority,
	deployment window etc.). This update policy should efficiently handle
	priorities, version dependencies, update conflicts and emergencies, in
	order the system to always be at a consistent state.
	
\end{itemize}
