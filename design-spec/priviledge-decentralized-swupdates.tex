\documentclass[11pt,a4paper]{article}
\usepackage[utf8]{inputenc}
\usepackage[margin=2.5cm]{geometry}

\usepackage{microtype}
\usepackage{mathpazo} % nice fonts
\usepackage{xcolor}
\usepackage[unicode=true,pdftex,pdfa,colorlinks=true]{hyperref}

\begin{document}

\hypersetup{
  pdftitle={Design specification of the Priviledge update mechanism},
  breaklinks=true,
  linkcolor={blue},
  citecolor={blue},
  urlcolor={blue},
  linkbordercolor={white},
  citebordercolor={white},
  urlbordercolor={white}
}

\title{
  Design specification of the Priviledge update mechanism\\
}

\author{
  Nikos Karagiannidis\\
  {\small \texttt{nikos.Karagiannidis@iohk.io}}\\
  \and
  Damian Nadales \\
  {\small \texttt{damian.nadales@iohk.io}}\\
  % TODO: add more authors if deemed appropriate.
}

\date{\today}

\maketitle

\begin{abstract}

\end{abstract}

\tableofcontents
\listoffigures
\listoftables

\section{Purpose}
\label{sec:purpose}

% What is the purpose of this document?
This document presents the design of a decentralized governance mechanism for
updating public blockchain systems.
% What is governance
In this context, governance refers to the set of rules by which the blockchain
evolves.
% When is governance decentralized?
We say that governance is decentralized when the community can use these rules
to determine the direction in which this blockchain evolves.

% TODO: should we mention here (or somewhre) how governance and the lifecycle of
% an update are related? We also need to define what an update is.

% Why people should care about the problems stated in this document?
A decentralized update mechanism is crucial for achieving a decentralized
blockchain.

% TODO: mention Voltaire, and how this document relates to it.

% Why people should care about this document? vs Target audience?
This document is intended for stakeholders interested in the implementation of a
decentralized update system for blockchains.

This document presents a breakdown of the requirements for achieving
decentralization of the update mechanism in a blockchain, and describes a design
that satisfies these requirements.

The topic of blockchain governance is difficult and broad, whereas the resources
available for tackling all the problems that governance poses are limited.
Therefore we identify important problems that were not solved in the present
document, but at the same time, we describe how the update mechanism described
in this document can accommodate any potential solutions to these problems.


\section{Requirements}
\label{sec:requirements}
% FROM: 2 Requirements for a decentralized software update system

This section presents the requirements for achieving decentralized governance of
a blockchain.

\subsection{Open}
\label{sec:open-participation}

\subsection{Democratic}
\label{sec:decentr-decis-making}

\subsection{Protocol driven}
\label{sec:protocol-driven}

\subsection{Transparent and auditable}
\label{sec:transp-audit}

\subsection{Secure}
\label{sec:secure}

\subsection{Performant and scalable}
\label{sec:performant-scalable}

\subsection{Metadata-driven}
\label{sec:metadata-driven}

\subsection{Consistent}
\label{sec:cons-update-logic}


\section{Design}
\label{sec:design}

% Hmmm, I'm thinking if we need to say something about the blockchain
% architecture we assume: for instance, we assume that we have a ledger
% components that will execute the update protocol somehow.

\subsection{Update types}
\label{sec:update-types}

% What is an update? How are updates related to the blockchain evolution and/or
% how are they related to governance.

% What aspects of a blockchain are we updating?

\subsection{The lifecycle of a decentralized update}
\label{sec:phases-an-update}
% FROM 3 The lifecycle of a decentralized software update

% Why do we need to have a governance system that captures the whole life cycle
% of an update? So for instance, why not supporting update activations only?

\subsection{Voting and delegation}
\label{sec:voting-delegation}
% FROM 4 Update governance

\subsubsection{Delegation}
\label{sec:delegation}

% When we describe how votes are counted we need to talk about delegation. Hence
% we must first talk about what delegation is.

\subsubsection{Voting}
\label{sec:voting}


\subsection{Threshold analysis}
\label{sec:threshold-analysis}
% FROM 6 Threshold analysis

\subsection{Ideation}
\label{sec:ideation}

\subsection{Implementation}
\label{sec:implementation}

\subsection{Approval}
\label{sec:approval}

\subsection{Activation}
\label{sec:activation}

% This seems like the right place to cite the Esorics paper. When we talk about
% the "enactment" process, we should mention what are the possible solutions. We
% should also mention the hard-fork combinator.

\subsection{Performance analysis}
\label{sec:performance-analysis}
% FROM performance considerations

\subsubsection{Impact on transaction throughput}
\label{sec:impact-trans-thro}

\subsubsection{Impact on processing time}
\label{sec:impact-proc-time}

\subsubsection{Impact on memory consumption}
\label{sec:impact-memory-cons}

\section{Satisfying the requirements}
\label{sec:satisfy-requ}

% How does the different aspects of the design satisfies the requirements.

% It seems that here we'd want to add the same subsections as in the
% requirements section.

\section{Implementation}
\label{sec:implementation-1}

Parts of the design described in this document was implemented and integrated to Cardano,
in a separate branch.

% Which parts were not implemented?

As a result of the implementation of the prototype:
\begin{itemize}
\item We developed a property testing framework
  \begin{itemize}
  \item based on IOHK's work
  \item that improved upon's IOHK's work
    \begin{itemize}
    \item faster generation and execution
    \item modular and simpler generators
    \end{itemize}
  \end{itemize}
\item Made the ledger layer of Cadano parametric on the update logic. This was
  merged onto master.
\item Ran benchmarks on a testnet that run the custom update mechanism which:
  \begin{itemize}
  \item confirmed that usage is linear on the number of participants as predicted
  \item latency is an aspect that needs to be carefully considered
  \end{itemize}
\end{itemize}

% The interested reader can check: deliverables, source code. Maybe an
% implementation report.

\section{Problems not addressed}
\label{sec:probl-not-addr}

% Which problems were not addressed? Why not?

% How can the prototype accommodate potential solutions to these problems.
          % - Using different voters set or voting weights.
          % - Incentives (it's an orthogonal, we see in principle nothing to
          %   accommodate).
          % - Spam protection: have a set of allowed submitters that is checked
          %   upon proposal submission (could be at commit time or at reveal
          %   time).

        % - How this prototype can accommodate a restriction in scope:
        %   - Having less phases.

        %   - Having a committee / set of genesis keys.

\section{Conclusions}
\label{sec:conclusions}



\bibliographystyle{acm}
\bibliography{references}

\end{document}
%%% Local Variables:
%%% mode: latex
%%% TeX-master: t
%%% End:
