\documentclass[11pt,a4paper]{article}
\usepackage[utf8]{inputenc}
\usepackage[margin=2.5cm]{geometry}

\usepackage{microtype}
\usepackage{mathpazo} % nice fonts
\usepackage{xcolor}
\usepackage[unicode=true,pdftex,pdfa,colorlinks=true]{hyperref}

\begin{document}

\hypersetup{
  pdftitle={Design specification of the Priviledge update mechanism},
  breaklinks=true,
  linkcolor={blue},
  citecolor={blue},
  urlcolor={blue},
  linkbordercolor={white},
  citebordercolor={white},
  urlbordercolor={white}
}

\title{
  Design specification of the Priviledge update mechanism\\
}

\author{
  Nikos Karagiannidis\\
  {\small \texttt{nikos.Karagiannidis@iohk.io}}\\
  \and
  Damian Nadales \\
  {\small \texttt{damian.nadales@iohk.io}}\\
  % TODO: add more authors if deemed appropriate.
}

\date{\today}

\maketitle

\begin{abstract}

\end{abstract}

\tableofcontents
\listoffigures
\listoftables

\section{Purpose}
\label{sec:purpose}

% What is the purpose of this document?
This document presents the design of a decentralized governance mechanism for
updating public blockchain systems.
% What is governance.
In this context, governance refers to the set of rules by which the blockchain
evolves.
% When is governance decentralized?
We say that governance is decentralized when the community can leverage on these
rules to determine the direction in which this blockchain evolves.

% TODO: should we mention here (or somewhere) how governance and the lifecycle of
% an update are related? We also need to define what an update is.

% Why people should care about the problems stated in this document?
A decentralized update mechanism is crucial for achieving a decentralized
blockchain.

% Why people should care about this document? What is the target audience?
This document is intended for stakeholders interested in the implementation of a
decentralized update system for blockchains.

% What does this document contains
This document features a breakdown of the requirements for achieving
decentralization of the update mechanism in a blockchain, and describes a design
that satisfies these requirements.

% Who funded this work?
The work presented here was funded by the PRIViLEDGE project~\cite{priviledge}.

% Scope of this work
The topic of decentralized blockchain governance is difficult and vast.
%
Therefore we identify important problems that were not solved within this
project, but at the same time, we show how the design described here can
accommodate any potential solutions to these problems.

% TODO: mention Voltaire, and how this document relates to it.

\section{Requirements}
\label{sec:requirements}
% FROM: 2 Requirements for a decentralized software update system

This section presents the requirements of an update mechanism for achieving
decentralized governance of a blockchain.

\subsection{Open}
\label{sec:open-participation}

In the absence of malicious actors, we could aspire that anybody who can
participate the blockchain protocol by submitting transactions can participate
in the update process. However allowing this would open the door for spam
attacks: if the barrier for submitting update proposals is as low as submitting
regular transactions, then we risk having a large number of bogus update
proposals being submitted that will overwhelm the human actors that have to
review these proposals.

In light of the risk presented in the preceding paragraph, we require that the
group of people that can participate in the update process should be as broad as
possible, but not broader. This means that:
\begin{itemize}
\item Restrictions on the group of people that can submit update proposals
  should be as loose as possible.
\item Any participant that can submit a common transaction should be able to
  vote on update proposals.
\end{itemize}

\subsection{Democratic}
\label{sec:decentr-decis-making}

All major decisions in the life cycle of a system update should be made by the
community, via a secure voting mechanism.

\subsection{Protocol driven}
\label{sec:protocol-driven}

The update protocol should be an on-chain process, which is automatically
enforced by the ledger rules of the blockchain.

\subsection{Transparent and auditable}
\label{sec:transp-audit}

Anybody should be able to answer \emph{when}, \emph{why}, and \emph{how} the
system has evolved the way it has. The systems' update history should be:
\begin{itemize}
\item open to everyone, and
\item stored in a tamper-free global ledger.
\end{itemize}

\subsection{Secure}
\label{sec:secure}

% TODO: what about other security aspects such as: preservation of ownership,
% resiliency to DoS attacks, replay protection, ensuring participation of an
% honest majority of stake-holders, ensuring liveness, etc.

\subsection{Performant and scalable}
\label{sec:performant-scalable}

The update protocol should have minimal impact on the:
\begin{itemize}
\item transaction throughput,
\item blockchain size, and
\item block processing time.
\end{itemize}
In addition, the impact the update protocol on the metrics above should be a
linear function of the number of participants.

\subsection{Metadata-driven}
\label{sec:metadata-driven}

Not all updates are the same. For instance some updates might be more urgent
than others, some updates might depend on a specific protocol version, some
updates might take longer than others, etc. Therefore, it should be possible to
specify proposal dependent features that the protocol must enforce.

\subsection{Consistent}
\label{sec:cons-update-logic}

The update protocol should define and automatically enforce a policy that is
consistent with the metadata of the update proposals. So for instance, the
update protocol should obey the priorities of proposals.

\section{Design}
\label{sec:design}

% Hmmm, I'm thinking if we need to say something about the blockchain
% architecture we assume: for instance, we assume that we have a ledger
% components that will execute the update protocol somehow.

% Updates are activated.

% Maybe say that we assume the blockchain time is split into epochs.

\subsection{Update types}
\label{sec:update-types}

% What is an update? How are updates related to the blockchain evolution and/or
% how are they related to governance.
Updates are the means by which the blockchain system and related ecosystem (e.g.
wallets, explorers) can evolve. We distinguish the following update types:
\begin{description}
\item[Protocol] These updates affect the consensus rules by which nodes agree on
  a common blockchain. Protocol updates fall in one of the following two
  categories:
  \begin{description}
  \item[Signaled] An update is said to be \emph{signaled} if the nodes must
    signal they readiness to run the new protocol. For instance, when a new
    protocol requires the nodes to download and install new software, they must
    have a way to determine when a sufficient majority of them have upgraded to
    the new version.
  \item[Non-signaled] An update is said to be \emph{non-signaled} if the nodes
    can move to the new protocol version without needing to know if other nodes
    are ready to move to said version as well. In this kind of updates we assume
    the nodes are already running a version of the software that can switch to
    the new protocol. An example of non-signaled protocol updates is
    \emph{parameters-update}. In a parameters-update, the nodes have already the
    logic for handling the parameters, and only the values of these parameters
    change after an update is activated.
  \end{description}
\item[Non-protocol] These updates do not affect the consensus rules. The current
  design does not restrict the nature of these updates. It is up to the ledger
  layer to react to these updates accordingly. They might include:
  \begin{itemize}
  \item updates to the software that runs on the blockchain nodes, which does
    not affect the protocol. Examples of such updates might include performance
    or usability improvements.
  \item transfer of funds. For instance an update proposal might involve funding
    the development of a certain feature.
  \item updates to components of the blockchain ecosystem.
  \end{itemize}
\end{description}

Note that, according to the preceding definition, an update that requires the
node to download and install new software to run a \emph{new protocol}, and that
\emph{also} updates protocol parameters is considered a signaled protocol
update.

% TODO: mention somewhere that protocol updates must be activated one at a time,
% whereas non-protocol updates can be approved in parallel.

\subsection{The lifecycle of a decentralized update}
\label{sec:phases-an-update}
% FROM 3 The lifecycle of a decentralized software update

% Why do we need to have a governance system that captures the whole life cycle
% of an update? So for instance, why not supporting update activations only?

\subsection{Voting and delegation}
\label{sec:voting-delegation}
% FROM 4 Update governance

\subsubsection{Delegation}
\label{sec:delegation}

% When we describe how votes are counted we need to talk about delegation. Hence
% we must first talk about what delegation is.

\subsubsection{Voting}
\label{sec:voting}

% Describe how votes are counted.

\subsection{Threshold analysis}
\label{sec:threshold-analysis}
% FROM 6 Threshold analysis

\subsection{Ideation}
\label{sec:ideation}

\subsection{Implementation}
\label{sec:implementation}

\subsection{Approval}
\label{sec:approval}

\subsection{Activation}
\label{sec:activation}

% This seems like the right place to cite the Esorics paper. When we talk about
% the "enactment" process, we should mention what are the possible solutions. We
% should also mention the hard-fork combinator.

\subsection{Performance analysis}
\label{sec:performance-analysis}
% FROM performance considerations

\subsubsection{Impact on transaction throughput}
\label{sec:impact-trans-thro}

\subsubsection{Impact on processing time}
\label{sec:impact-proc-time}

\subsubsection{Impact on memory consumption}
\label{sec:impact-memory-cons}

\section{Satisfying the requirements}
\label{sec:satisfy-requ}

% How does the different aspects of the design satisfies the requirements.

\subsection{Open}
\label{sec:sat-open}

This design does not address the definition of minimal requirements for proposal
submissions that will prevent spam attacks.

This design addresses the requirement that any participant that can submit a
common transaction should be able to vote on update proposals, by including
defining a new \emph{vote} transaction type. Furthermore, the fee that these
vote-transactions pay are calculated in the same way as monetary transactions.
In this way, stakeholders can vote directly by submitting a vote, or indirectly
by delegating their voting rights to experts.

\subsection{Democratic}
\label{sec:sat-decentr-decis-making}

This design addresses the requirements that the community can vote on updates,
by defining an on-chain voting mechanism in which participants either:
\begin{itemize}
\item vote directly by submitting a vote to the blockchain in the same way they
  do for regular transactions, or
\item delegate their vote to experts who will vote on their behalf.
\end{itemize}

\subsection{Protocol driven}
\label{sec:sat-protocol-driven}

The update mechanism here described can be run as part of the ledger rules of a
blockchain.

\subsection{Transparent and auditable}
\label{sec:sat-transp-audit}

By making the update proposals and votes part of the transactions that are
included in blocks, we ensure that the system updates history is stored in the
immutable ledger that is the blockchain.

\subsection{Secure}
\label{sec:sat-secure}

% TODO: is delegation to experts part of the security of the protocol?
% Delegation to experts:
%
% - makes participation more likely (although this depends on the experts' incentives)
%
% - put's important decisions in the hands of the people that have the technical
%   knowledge
%

\subsection{Performant and scalable}
\label{sec:sat-performant-scalable}

Our back-of-the envelope calculations showed that the update protocol has an
impact on the system throughput that is linear on the number of participants.
These numbers were confirmed by the measurements we took after running several
experiments on a testnet.

Our asymptotic complexity analysis and micro-benchmarks showed that the impact
on system's performance is also linear on the number of participants.

\subsection{Metadata-driven}
\label{sec:sat-metadata-driven}

We have define an update protocol where the submitter of a proposal can specify
(when applicable):
\begin{itemize}
\item its priority,
\item its dependency,
\item the duration of its voting period,
\item the duration of its deployment window.
\end{itemize}

\subsection{Consistent}
\label{sec:sat-cons-update-logic}

The activation protocol in Section~\ref{sec:activation} defines a queuing
mechanism that ensures that proposals are activated in the order prescribed by
their priorities. Furthermore, this queuing mechanism ensures a candidate
for activation always supersedes the current protocol version.

The voting period of a proposal submitted to either the ideation or approval
phases is taken from the proposal's metadata.
% TODO: we should make sure that we cap this voting period, and mention this
% exception here.
Similarly, the deployment window, in the case of consensus-impacting updates, is
also determined by the proposal's metadata.

\section{Implementation}
\label{sec:implementation-1}

% Mention that we have an implementation of the concepts presented in this
% document, without giving much details about it. The interested reader can look
% at the code (and maybe we'll have time to document this design).
Parts of the design described in this document was implemented and integrated to Cardano,
in a separate branch.

% Which parts were not implemented?
Expert pools and delegation was not implemented in the prototype. These aspects
are orthogonal to the update logic that was implemented in the prototype, and as
such, they can be added later.

As a result of the implementation of the prototype:
\begin{itemize}
\item We developed a property testing framework, which improves upon previous
  work by IOHK.
\item We made the ledger layer of Cadano parametric on the update logic. This
  was merged onto master, and as a result it is now easier to swap Cardano's
  update mechanism.
\item We ran benchmarks on a testnet. The results we obtained from these
  benchmarks confirmed that usage is linear on the number of participants as
  predicted. On the other hand, we observed a quadratic increase in the
  transaction latency. In our benchmarks we used a very short voting period of
  10 minutes, and we reproduced the worst-case and unrealistic assumption that
  every participant would vote at the same time. Nevertheless, further analysis
  is needed for a production ready update protocol to determine if safeguard
  measures against high transaction latency are required.
\end{itemize}

% The interested reader can check: deliverables, source code. Maybe an
% implementation report.
The reader interested in the design and implementation of the prototype can
check the PRIViLEDGE deliverables~\cite{priviledge_d11, priviledge_d12,
  priviledge_d41}, as well as the implementation documentation and source
code~\cite{dsu-repo2021}.

\section{Related work}
\label{sec:related-work}

\section{Problems not addressed}
\label{sec:probl-not-addr}

% Which problems were not addressed? Why not?

% How can the prototype accommodate potential solutions to these problems.
          % - Using different voters set or voting weights.
          % - Incentives (it's an orthogonal, we see in principle nothing to
          %   accommodate).
          % - Spam protection: have a set of allowed submitters that is checked
          %   upon proposal submission (could be at commit time or at reveal
          %   time).

        % - How this prototype can accommodate a restriction in scope:
        %   - Having less phases.

        %   - Having a committee / set of genesis keys.

\section{Conclusions}
\label{sec:conclusions}

\bibliographystyle{acm}
\bibliography{references}

\end{document}
%%% Local Variables:
%%% mode: latex
%%% TeX-master: t
%%% End:
