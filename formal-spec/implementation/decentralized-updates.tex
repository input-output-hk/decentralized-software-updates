\documentclass[11pt,a4paper]{article}
\usepackage[utf8]{inputenc}
\usepackage[margin=2.5cm]{geometry}

\usepackage{microtype}
\usepackage{mathpazo} % nice fonts
\usepackage{amsmath}
\usepackage{amssymb}
\usepackage{latexsym}
\usepackage{mathtools}
\usepackage{stmaryrd}
\usepackage{extarrows}
\usepackage{slashed}
\usepackage[unicode=true,pdftex,pdfa,colorlinks=true]{hyperref}
\usepackage{xcolor}
\usepackage[capitalise,noabbrev,nameinlink]{cleveref}
\usepackage{float}
\floatstyle{boxed}
\restylefloat{figure}
\usepackage{listings} % for code blocks.
%%
%% Package `semantic` can be used for writing inference rules.
%%
\usepackage{semantic}
%% Setup for the semantic package
\setpremisesspace{20pt}

\usepackage{pgf}
\usepackage{pgfplots}
\usepackage{tikz}
\usepackage{tikz}
\usepackage{hyperref}
\usetikzlibrary{arrows,automata, decorations.pathreplacing, positioning, arrows.meta, calc, shapes}
% For drawing simple diagrams involving arrows between LaTeX symbols
\usepackage{tikz-cd}

\newtheorem{definition}{Definition}
\newtheorem{property}{Property}

\newcommand{\size}[1]{\ensuremath{\left| #1 \right|}}
\newcommand{\trans}[2]{\ensuremath{\xlongrightarrow[\textsc{#1}]{#2}}}
\newcommand{\transtar}[2]{\ensuremath{\xlongrightarrow[\textsc{#1}]{#2}\negthickspace^{*}}}

\newcommand{\autact}[3]{\ensuremath{\scalebox{0.75}{$
    \begin{array}{l l}
      & #1\\
      ; & #2\\
      ; & #3
    \end{array}
$}}}

\newcommand{\nnote}[1]{{\color{blue}\small Nikos: #1}}


\lstdefinestyle{mystyle}{
%    backgroundcolor=\color{backcolour},
    commentstyle=\color{codegreen},
    keywordstyle=\color{blue},
%    numberstyle=\tiny\color{codegray},
    stringstyle=\color{codepurple},
    basicstyle=\ttfamily\footnotesize,
    breakatwhitespace=false,
    breaklines=true,
    captionpos=b,
    keepspaces=true,
    numbers=left,
    numbersep=5pt,
    showspaces=false,
    showstringspaces=false,
    showtabs=false,
    tabsize=2
}

\lstset{style=mystyle}

% Use this when working in particular chapters.
%
% \includeonly{update-protocol}

\begin{document}

%%%%%%%%%%%%%%%%%%%%%%%%%%%%%%%%%%%%%%%%%%%%%%%%%%%%%%%%%%%%%%%%%%%%%%%%%%%%%%%%
%% Front matter
%%%%%%%%%%%%%%%%%%%%%%%%%%%%%%%%%%%%%%%%%%%%%%%%%%%%%%%%%%%%%%%%%%%%%%%%%%%%%%%%

\hypersetup{
  pdftitle={A Formal Specification of an Update Mechanism for Cardano},
  breaklinks=true,
  bookmarks=true,
  colorlinks=false,
  linkcolor={blue},
  citecolor={blue},
  urlcolor={blue},
  linkbordercolor={white},
  citebordercolor={white},
  urlbordercolor={white}
}

\title{
  A Formal Specification of an Update Mechanism for Cardano\\
}

\author{
  Nikos Karagiannidis\\
  {\small \texttt{nikos.Karagiannidis@iohk.io}}\\
  \and
  Michele Ciampi \\
  {\small \texttt{micheleciampi1990@gmail.com}}\\
  \and
  Damian Nadales \\
  {\small \texttt{damian.nadales@iohk.io}}\\
}

\date{\today}

\maketitle

\begin{abstract}

\end{abstract}

%%%%%%%%%%%%%%%%%%%%%%%%%%%%%%%%%%%%%%%%%%%%%%%%%%%%%%%%%%%%%%%%%%%%%%%%%%%%%%%%
%% End: Front matter
%%%%%%%%%%%%%%%%%%%%%%%%%%%%%%%%%%%%%%%%%%%%%%%%%%%%%%%%%%%%%%%%%%%%%%%%%%%%%%%%

\tableofcontents
\listoffigures

\section{Introduction}
Software updates are everywhere. The most vital aspect for the sustainability of any software system is its ability to effectively and swiftly adapt to changes; one basic form of which are software updates. Therefore the adoption of software updates is at the heart of the lifecycle of any system and blockchain systems are no exception. Software updates might be triggered by a plethora of different reasons: change requests, bug-fixes, security holes, new-feature requests, various optimizations, code refactoring etc.

%Typically, the main driver for a change will be a change request, or a new feature request by the user community. These type of changes will be planned for inclusion in some future release and then implemented and properly tested prior to deployment into production. In addition, another major source of changes is the correction of bugs (i.e., errors) and the application of security fixes. These usually produce the so-called \say{hot-fixes}, which are handled in a totally different manner (i.e., with a different process that results into different deployment speed) than change requests. In addition, depending on the severity level of the problem, or the priority (i.e., importance) of the change request there are different levels of speed and methods for the deployment of the software update into production.

More specifically, for blockchain systems, a typical source of change are the enhancements at the consensus protocol level. There might be changes to the values of specific parameters (e.g., the maximum block size, or the maximum transaction size etc.), changes to the validation rules at any level (transaction, block, or blockchain), or even changes at the consensus protocol itself. Usually, the reason for such changes is the reinforcement of the protocol against a broader scope of adversary attacks, or the optimization of some aspect of the system like the transaction throughput, or the storage cost etc.
%Finally, there are also more radical changes, which are usually caused by the introduction of new research ideas  and the advent of new technology, which becomes relevant. These type of changes usually introduce new concepts and are not just enhancements and thus trigger a major change to the system.
%\paragraph{Context of this paper.}
In this paper, our focus is on the software update mechanism of stake-based blockchain systems. We depart from the traditional centralized approach of handling software updates, which is the norm today for many systems (even for the ones that are natively decentralized, like permission-less blockchain systems) and try to tackle common software update challenges in a decentralized setting. We consider the full lifecycle of a software update, from conception to activation and propose decentralized alternatives to all phases. Essentially, we introduce a \emph{decentralized maintenance} approach for stake-based blockchain systems.

%\paragraph{Context of this paper.} In this paper, our focus is on the update mechanism of permissionless stake-based blockchain systems. We try to overcome known shortcomings of today's update methods for blockchain systems and propose a logical architecture for an update mechanism for stake-based ledgers. Our architecture covers all the components of a typical blockchain system and demonstrates how can such a mechanism be smoothly incorporated into each layer of the blockchain system. We describe in detail the various enhancements and changes in the existing components and then proceed into describing new components such as the governernance component, or the update-logic component. Our aim is to cover all aspects of software updating, from update proposal submission to update deployment and activation. Finally, we implement our logical architecture into a prototype update system for the Cardano blockchain. We then discuss various implementation issues and the prototype architecture.


\paragraph{Problem Definition.}
Traditionally, software updates for blockchain systems have been handled in an ad-hoc, centralized manner: somebody, often a trusted authority, or the original author of the software, provides a new version of the software, and users
download and install it from that authority's website. Even if the system follows an open source software development model, and therefore an update can potentially be implemented by anyone, the final decision of accepting, or rejecting, a new piece of code is always taken by the main maintainer(s) of the system, who essentially constitutes a central authority. Even in the case where the community has initially reached consensus for an update proposal (in the form of an \emph{Improvement Proposal} document), through the discussion that takes place in various discussion forums, still it remains an informal, \say{social} consensus, which is not recorded as an immutable historical event in the blockchain and the final decision is always up to the code maintainer. Moreover, the authenticity and safety of the downloaded software is usually verified by the digital signature of a trusted authority, such as the original author of the software.

\paragraph{Our contributions.}
We put forth a novel mechanism for realizing software updates. In our proposed scheme, an update proposal is possible to be submitted by anyone who can submit a transaction to the blockchain. The decision of which update proposal will be applied and which will not is taken collectively by the community and not centrally. Thus, the roadmap of the system is decided jointly. Moreover, this process is no longer an informal discussion process, but part of an \emph{update protocol}. All relevant events generated are stored within the blockchain itself, and thus recorded in the immutable update history of the system. Moreover, the role of the code maintainer, who used to take decisions on the correctness of the submitted new code and guarantees the validity of the downloaded software, is replaced by the stakeholders' community.

In the context of software updates for public stake-based blockchain systems, we introduce the capability to take stake-based decisions
%\mnote{I am not sure that we should talk about stake majority-based decision. Indeed, we later discuss the possibility of setting the threshold (that defines whether a proposal should be accepted or not) to be anything greater than $52\%$}
 based on: a) the software update priority, b) the correctness of the new code, c) the maintenance of the code base and d) the authenticity and safety of the downloaded software. We introduce the problem of then activating the changes on the blockchain without risking a chain split
%, as part of the consensus protocol and recorded on-chain, while at the same time fulfilling software dependencies requirements, resolving version conflicts, enabling different update policies based on update metadata and avoiding chain splits when updates are activated
 as the \emph{decentralized software updates problem}, and put forth the first solution to this problem. In addition, we investigate how to enable different update policies based on the software update context (i.e., update metadata) and at the same time fulfill software dependency requirements and resolve conflicts (cf. Appendix \ref{appxupdlogic}).
%\mnote{The latest might become on of the main focus of the paper. We should highlight that.}

%\paragraph{Problem Definition.}  Traditionally, software updates have been handled in an ad-hoc, centralized manner: Somebody, often
%a trusted authority, or the original author of the software, provides a new version of the software, and users
%download and install it from that authority's website. However, this approach is clearly not decentralised, and
%hence jeopardizes the decentralized nature of the whole system: In a decentralized software update mechanism,
%proposed updates can be submitted by anyone (just like anyone can potentially create a transaction in a blockchain).
%The decision of which update proposal will be applied and which wont, is taken collectively by the community
%and not centrally. Thus, the road-map of the system is decided jointly. Moreover, there is no central code
%repository, nor there is some main software maintainer, who decides on the code. All versions of the code
%are distributed and only local copies exist, in the same manner that the ledger of transactions is distributed in
%blockchain. The decentralized software update system must reach a consensus, as to what version is the current
%one (main branch) and which code branch will be merged into main. Finally, the decentralized software update
%system guarantees the authenticity and safety of the downloaded software, without the need to have some central authority
%to sign the code, in order to be trusted


%The traditional way of handling software updates is neither decentralized nor secure
%\begin{itemize}
%\item No standard way to propose updates
%\item Not a decentralized and democratic way to reach at a consensus on update priorities. Only \enquote{social consensus} is reached via social media. This is unstable and prone to chain splits.
%\item No essential auditing and verifiability of the agreement
%\item No standard way to record the immutable history of all these events (proposals,agreement, \item activation of updates)
%\item No standard method for security guarantees for the software installed
%\item No standard way to resolve conflicts and respect dependencies
%\item No standard Update metadata
%\item One-size-fits-all for all \enquote{types} of changes (bugs,CRs)
%\item Hard-forks are the norm
%\end{itemize}

\import{./}{related_work.tex}
\paragraph{Goal of the paper.} In this paper we propose a secure software update mechanism that enables a decentralized approach to the blockchain software updates problem. We examine all phases in the lifecycle of a software update and propose practical decentralized alternatives that can be adopted in the real world. These alternatives substitute any \emph{central authority} with the \emph{stakeholders' community}. In order to enable this \emph{decentralization} of software updates, we exploit existing primitives that we combine in order to form a novel decentralized software updates protocol. From a security perspective, we formally define what is a secure activation of changes on a blockchain and prove the security of our protocol with respect to this definition.  Our protocol ensures that:
%...A safe and secure decentralized software update system. %In other words, a system that will ensure that:
a) any stakeholder will always be able to submit an update proposal to be voted by the stakeholders' community, b) an update proposal that is not approved by the stakeholders
%\mnote{Also here, let's talk about a threshold $t$ greater than $51\%$ instead of stake majority since this is what we do}
 will never be applied, c) an update proposal that it is approved by the stakeholders will be eventually applied, d) an update proposal that the stakeholders decide has a higher priority than some other proposal will take higher priority, e) downloaded software is authentic and safe, and finally f) it protects against chain splits during activation of the proposed updates.

\paragraph{Outline of the paper.} In Section~\ref{lifecycle} we present our decentralized approach for the lifecycle of a software update. Section~\ref{secureupdate} defines what a secure software update mechanism is and proves the security of our protocol. In the appendix, we contrast with the centralized approach to the software update lifecycle, provide more details for the voting and delegation mechanisms of our protocol and we discuss our proposal for a metadata-driven software update mechanism.
%\noindent\textbf{Outline of the paper}

\section{Requirements for the software update mechanism} \label{sec:requirements}

\subsection{Functional requirements} \label{sec:func-reqs}
\begin{enumerate}
\item \textbf{Updates are open to the community.} \label{req:upd-open} Any
  stakeholder should be able to propose a software update to be voted by the
  stakeholders' community.

\item \textbf{Updates utilize the blockchain} \label{req:upd-blockchain} Update
  events are stored in the blockchain as immutable events.

\item \textbf{Decentralized Governance} \label{req-dec-gov} The stakeholders'
  community is the only authority to decide on the approval, or rejection, of
  the proposed software updates.

\item \textbf{Secure Activation} \label{req-sec-act} Upon activation of a
  software update the update mechanism guarantees a secure transition to the
  updated ledger; a transition in which:
  \begin{itemize}
  \item The security assumptions of the updated ledger will hold (e.g., there
    will be a sufficient percent of upgraded honest stake)
  \item The updated ledger successfully incorporates the old ledger, i.e., the
    state of the old ledger will be successfully moved into the new ledger.
  \item There will be no chain splits due to the activation.
\end{itemize}

\item \textbf{Conflict Resolution.}\label{req:conflict-res} Two or more software
  updates that have conflicting changes will not be applied both, but a conflict
  resolution will take place.

\item \textbf{Update Dependencies.}\label{req:update-dep} A software update will
  never be applied if its dependency requirements are not met

\item \textbf{Extensible voting period.} \label{req:ext-vper} The period
  required for approving a software update is not fixed but can be extended by
  the stakeholders' community. In other words, revoting is supported.

\end{enumerate}

\subsection{Non-functional requirements} \label{sec:non-func-reqs}

\begin{enumerate}
\item \textbf{Transaction Throughput}
  \label{req:tr-throughput}
  The ledger transaction throughput is not significantly impacted by the
  presence of software updates (see section \ref{sec:measurements}).

\item \textbf{Blockchain Size} \label{req:blockchain-size} The blockchain size
  is not significantly impacted by the presence of software updates (see section
  \ref{sec:measurements}).

\item \textbf{Update Time to Activation}\label{req:update-thr} The time to
  activation of a software update reduces linearly in the number of users
  running the update protocol (see section \ref{sec:measurements}).

\item \textbf{Processing Time}
  \label{req:proc-time}
  The processing time of a node due to the execution of the update protocol is
  not significantly increased.

\end{enumerate}

%%% Local Variables:
%%% mode: latex
%%% TeX-master: "decentralized-updates"
%%% End:

\section{Properties} \label{sec:cand-properties}

\subsection{Global properties}

\begin{property}[Software Update Proposal]\label{prop:submissions-for-all}
  A successfully submitted proposal always has an author who belongs to the set
  of stakeholders.
\end{property}

\paragraph{Classify:} Property-based testing. Property fulfilling requirement
\ref{req:upd-open}

\begin{property}[Updates Utilize Blockchain]\label{prop:utilize-blockchain}
  The number of generated update events equals the on-chain immutable update
  events.
\end{property}

\paragraph{Classify:} Property-based testing. Property fulfilling requirement
\ref{req:upd-blockchain}

\begin{property}[Non Approved Software Updates]\label{prop:nonapproved-sus}
  A software update that is not approved by the stakeholders' community will
  never be applied.
\end{property}

\paragraph{Classify:} Property-based testing. Property fulfilling requirement
\ref{req-dec-gov}

\paragraph{how to test}
Property-based testing: Every approved software update (SIP or UP) must have
reached a stake in favor result above the threshold $\tau_V$.

\begin{property}[Approved Software Updates]\label{prop:approved-sus}
  An approved proposal cannot be blocked. Based on the assumption that given
  sufficient time, an honest party will eventually upgrade to an approved
  software update and that the software update is always available for
  downloading, then a software update that has been approved by the
  stakeholders' community will be eventually applied.
\end{property}

\paragraph{Classify:} Property-based testing. Property fulfilling requirement
\ref{req-dec-gov}

\paragraph{how to test}
Every UP that has met its activation condition will be activated.

\begin{property}[Secure Activation]\label{prop:updated-ledger-security}
  Our update mechanism guarantees a secure activation according to the
  requirement \ref{req-sec-act}.
\end{property}

\paragraph{Classify:}
Mathematical proof. Alternatively, Property-based testing. Property fulfilling
requirement \ref{req-sec-act}.

\paragraph{how to test}
When the updated ledger obtains the first stable block (genesis block), then for
any two parties $P_1$ and $P_2$, the stable part of $P_1$ will be a prefix of
the updated ledger of $P_2$. Moreover, at this moment, the activation condition
(based on the adoption threshold $\tau_A$ and the safety lag) will hold, which
means that the security assumptions of the new ledger will hold.

\begin{property}[Conflict Resolution]\label{prop:conflict-res}
  There is no trace where two or more conflicting updates are in one of the
  states: Active, Approved or To-Be-Activated at the same time.
\end{property}

\paragraph{Classify:}
Property-based testing. Property fulfilling requirement \ref{req:conflict-res}

\begin{property}[Update Dependencies]\label{prop:update-dep}
  There is no trace where an update whose dependency requirements are not met
  and it is in one of the states Active, Approved or
  To-Be-Activated.
\end{property}

\paragraph{Classify:} Property-based testing. Property fulfilling requirement
\ref{req:update-dep}

\begin{property}[Extensible Voting Period]\label{prop:dur-vot-period}
  It is possible to extend the voting period once it has ended.
\end{property}

\paragraph{Classify:} Property-based testing. Property fulfilling requirement
\ref{req:ext-vper}

%\subsection{Consensus Protocol Properties}
%
%\begin{property}[Bootstrapping]\label{prop:bootstrap}
%  The update mechanism leverages the consensus protocol of the blockchain
%  itself.
%\end{property}
%
%\begin{property}[Resiliency]\label{prop:resiliency}
%  The resiliency of the consensus protocol (expressed as the fraction ($t/n$
%  where $t$ is the adversary stake and $n$ is the total stake) of misbehaving
%  parties a protocol can tolerate) is not affected by the update mechanism
%\end{property}
%
%\begin{property}[Running Time]\label{prop:run-time}
%  The worst number of rounds by which honest parties terminate the consensus
%  protocol (i.e., reach a consensus) is not affected.
%\end{property}
%
%\begin{property}[Communication Complexity]\label{prop:comm-complex}
%  The worst total number of bits/messages communicated during a consensus
%  protocol run is not affected.
%\end{property}

\subsection{Ideation phase properties}

\begin{property}[Uniqueness of SIPs]\label{prop:unique-sip}
  Submitted SIPs in the trace are unique.
\end{property}

\begin{property}[Valid Ownership of SIPs]\label{prop:owner-sip}
  SIPs can be submitted only by stakeholders of the ledger.
\end{property}

\begin{property}[Validity of Revealed SIPs]\label{prop:reveal-valid}
  Every Revealed SIP must correspond to one and only stable submitted SIP.
\end{property}

\begin{property}[Validity of a SIP]\label{prop:sip-valid}
  Every valid SIP upon revealing must comprise a valid set of metadata and an
  active URL pointing to an off-chain stored SIP bundle, whose structure
  conforms to the SIP structure guideline.
\end{property}

\begin{property}[Validity of Active SIPs]\label{prop:active-sip-valid}
  Every active SIP must correspond to one and only stable revealed SIP, whose
  voting period end has not yet arrived.
\end{property}

\begin{property}[Validity of SIP votes]\label{prop:sip-votes-valid}
  Every vote for a SIP must correspond to one and only valid active SIP (see
  property \ref{prop:active-sip-valid}). Therefore the slot of the vote cannot
  exceed the slot of the voting period end of the SIP in question.
\end{property}

\begin{property}[Number of Votes]\label{prop:num-votes-sip}
  To every active SIP correspond zero or more votes.
\end{property}

\begin{property}[SIP Tally]\label{prop:sip-tally-valid}
  A valid SIP tally refers to one and only SIP whose voting period has ended at
  least 2k slots ago and counts only valid votes (see property
  \ref{prop:sip-votes-valid}).
\end{property}

\begin{property}[SIP Tally Results]\label{prop:sip-tally-result}
  A SIP in the trace that has been stably revealed, can be either Active,
  Approved, Rejected or Expired.
\end{property}

\begin{property}[SIP Revoting]\label{prop:sip-revote}
  A SIP in the trace cannot be revoted more than $prvNoQuorum + prvNoMajority$
  times.
\end{property}

\subsection{Implementation phase properties}

\begin{property}[Approved SIPs]\label{prop:sip-approved-valid}
  Every approved SIP corresponds to a set of valid votes (see property
  \ref{prop:sip-votes-valid}) that if tallied will yield a stake in favor above
  the voting threshold $\tau_V$.
\end{property}

\subsection{Approval phase properties}

\begin{property}[Correctness and accuracy]\label{prop:corr-accuracy}
  The UP implements correctly (i.e., without bugs) and accurately (i.e., with no
  divergences) the changes described in the corresponding voted SIP, as per the
  approval of the stakeholders' community.
\end{property}

\begin{property}[Continuity]\label{prop:continuity}
  Nothing else has changed beyond the scope of the corresponding SIP and
  everything that worked in the base version for this UP, it continues to work,
  as it did (as long as it was not in the scope of the SIP to be changed), as
  per the approval of the stakeholders' community.
\end{property}

\subsection{Activation phase properties}

\begin{property}[Sequential Activation of UPs]\label{prop:seq-activation}
  Only one UP can be activated at a time.
\end{property}

\begin{property}[Activation Priorities of UPs]\label{prop:activation-priorities}
  An UP that the stakeholders decide has a higher priority than some other
  proposal will take higher priority in the activation sequence.
\end{property}

\begin{property}[Authenticity and safety]\label{prop:auth-safety}
  Downloaded software is safe as per the approval by the stakeholders'
  community. Moreover by downloading it, one downloads the original authentic
  code that has been submitted and approved in the first place.
\end{property}

\begin{property}[Denial of Activation]\label{prop:doactivation}
  Assuming that given sufficient time an honest party will always upgrade to an
  approved UP, then an approved UP cannot be blocked from activation by the
  adversary.
\end{property}

\begin{property}[Honest Stake Majority]\label{prop:honest-majority}
  Upon activation of a UP the updated ledger will consist of sufficient honest
  to run securely the consensus protocol.
\end{property}

\begin{property}[Chain Split]\label{prop:chain-split}
  Upon activation of a UP the system is protected from chain splits.
\end{property}


%
% a) a software update that has not achieved an honest stake majority approval,
% will never be allowed to pass to the next phase in the SU lifecycle and
%
% b) a software update that has achieved an honest stake majority approval, will
% eventually be allowed to pass to the next phase in the SU lifecycle.

%%% Local Variables:
%%% mode: latex
%%% TeX-master: "decentralized-updates"
%%% End:

\section{Update protocol}

\subsection{Activation phase}
\label{sec:activation-phase}

\emph{Activation} is the phase in the life-cycle of an update proposal where it
might come into effect.
%
This phase follows the approval phase where the submitted implementations are
reviewed by the expert pools (who are delegated by the community).
%
Through a voting process the experts decide whether an implementation of a
software update will proceed to the activation phase or whether it will be
rejected.

An update can be of four types:
\begin{description}
\item[Parameters] which are updates that change the system parameters.
\item[Consensus] which are updates that change the consensus rules. For instance
  changes in the transaction or block validation rules, or changes in the fork
  resolution rules. As an extreme example, even the underlying consensus
  protocol could be changed (for instance, going from Ouroboros classic to
  Ouroboros BFT).
\item[Application] which are updates of applications in the Cardano ecosystem:
  nodes, wallets, explorers, etc. Application updates do not change the
  blockchain protocol.
\item[Cancellation] which are special kind of update proposals that can be used
  to cancel other \emph{approved} proposals\footnote{For non-approved proposals
    the expert pools can simply reject said update}.
\end{description}

Application updates do not require synchronization among the nodes of the
blockchain. Since this update does not affect consensus or might not even be
related to a software update for a node, the community can upgrade to a new
version of a given application as soon as it is (stably) approved.

A protocol-parameters update comes into effect \emph{at the beginning of a new
  epoch}. The nodes do not need to install new software, since the current
running version can update the parameters as prescribed in an approved update
proposal.

A consensus-update requires that the parties \emph{synchronize} prior to its
activation to avoid a chain split due to lack of consensus. In addition, before
the activation, we need to make sure that certain security assumptions hold,
like the fact that enough honest stake has upgraded.
%
To this end, the activation phase synchronizes participants to make sure that at
least a certain portion of the honest stake will activate the update at the same
time, which is a necessary condition to ensure that the upgraded ledger will be
secure.
%
In this context, activation of an update means that the protocol version that it
specifies becomes the current version of the blockchain.

The stake that can endorse is the stake delegated to stake pools, which is the
stake that is considered for selecting block producers.

Note that a consensus-update can also specify parameters update. In this case
the new software needs to be installed and run on the nodes, and the new
software must apply, at the beginning of an epoch, the parameters updates that
the update proposal dictates.

Additionally, consensus update must specify where the code and binaries that
run the new protocol can be obtained.

All parameters and consensus proposals must specify:
\begin{itemize}
\item The version that they supersede. The proposal can only be applied to this
  version.
\item The (hash of the) proposal they supersede. Update proposals in the
  approval phase can have the same version, so proposals need to disambiguate
  the proposal they supersede. This allows us to avoid a situation in which a
  proposal depends on a given version, but also (implicitly) on a particular
  implementation of that version. If we do not make this dependency implicit,
  then we risk having an update being applied on the wrong version.
  %
  We cannot on the other hand, enforce uniqueness of versions in update
  proposals, since we risk having malicious actors organizing a denial of
  service attack that prevents honest proposals from entering the system.
  %
  Note that having a proposal specify the proposal it supersedes require that we
  start with a Genesis (implementation) proposal.
\end{itemize}

When a parameters or consensus proposal is approved\footnote{Cancellation and
  application update proposals come into effect immediately after being
  approved.} it enters a \emph{priority queue} where it waits to be activated.
The priority of the queue is determined by the update proposal's versions. A
proposal with lower version will activate \textbf{before} a 
proposal with a
higher version.
%
We call the proposals in the activation queue the \emph{approved proposals}.

Proposals \textbf{must increase} the current \emph{protocol-version} that the
chain supports. The protocol version consists of two components:
\begin{enumerate}
\item The major version, which is used to specify hard-forks.
\item The minor version, which is used to specify soft-forks.
\end{enumerate}

The update system guarantees that the protocol versions can only strictly
increase. In this way, the update system can rely on the update proposal's
protocol-version and its declared predecessor version to:
\begin{itemize}
\item Determine priorities: a lower version (higher priority) is activated 
before a higher version (lower priority) one; specifying the version gives us a 
simple way to specify urgency.
\item Resolve conflicts: a proposal declares \textbf{one and only one} version
  that it supersedes (its predecessor). Thus a proposal is in conflict with all
  others except the one that it supersedes.
\item Check dependencies: a proposal depends only on the one it supersedes.
\end{itemize}
This design decision requires:
\begin{itemize}
\item The implementers to pick exactly one version their update is compatible
  with, and determine which version their update should receive.
\item The experts to check the version that an update has, and the compatibility
  with the version it supersedes.
\end{itemize}

Protocol-versions of update proposals do not need to be unique. 
%However, if two
%proposals with the same version are in the approval phase in a given slot, then
%the experts can \textbf{vote for at most one of them}. This guarantees that at
%any slot only one proposal per-version can make it to the activation phase
%since:
%\begin{itemize}
%\item The expert pools can vote for one proposal with the same version at the
%  same slot.
%\item A majority of stake is required to approve the proposal.
%\nnote{This is not true. The requirement is to have stake in 
%favor above the threshold, which might be below majority i.e., 51\%. so both 
%proposals might exceed the threshold ang get approved}

%\end{itemize} 

If two proposals with the same version are in the approval phase and both are 
approved, then the last one entering the activation phase, will survive (see 
section \ref{sec:cancellations} on cancellations). In the unlikely case, where 
the two proposals coincide in their voting period ends and thus are approved at 
the very same slot, then we resolve this conflict by choosing the one with the 
higher stake in favor to enter the activation queue. In the even more unlikely 
case, where both have the same stake in favor, then we choose the proposal with 
the greatest id (proposal hash) to enter the activation queue.

The approved proposal at the front of the activation queue will enter the
\emph{endorsement period}, if it supersedes the current version (see section 
\ref{sec:entering-the-endorsement-phase} ).
%
There can be \emph{at most one proposal} in the endorsement period at any time.
%
This is to avoid the stake of the stakeholders being split among competing
proposals during endorsement, and thus, when changes take effect.
%
We call a proposal that is in the endorsement period the \emph{candidate
  proposal}.

If the new candidate proposal is a consensus-update, the block producers can
start \emph{endorsing} it. An endorsement is a special metadata field in the
block, and can be set only by the block producers. An endorsement signals the
fact that a node has downloaded and installed the software that implements the
update.
%
If the proposal gathers enough endorsements, then it can be
\emph{scheduled-for-activation}.
%
If the candidate proposal does not need endorsements, i.e. it is a protocol
parameters update, it becomes immediately scheduled.

A scheduled proposal \textbf{waits} to be activated \textbf{till the first slot
  of the next epoch}. In case of application updates, the proposal does not need
to be activated. The ledger simply records its approval.

The next sub-sections describe the details of the activation protocol outlined
above.

\subsubsection{Entering the endorsement period}
\label{sec:entering-the-endorsement-phase}

An update proposal can enter the endorsement period if two conditions are met:
\begin{enumerate}
\item It is at the front of the priority queue (i.e. it is has the lowest
  version among the proposals queue).
\item The version and proposal hash that the proposal declares to supersede must
  be \textbf{the same} as the current version.
\end{enumerate}
Once in the endorsement period, \textbf{it might return to the queue} if a new
update proposal is approved and ends up in front of the priority queue (which
means that the newly approved proposal has a lower version than the current
candidate).

\subsubsection{Endorsements of consensus updates}
\label{sec:endorsemnts}

Only a consensus-update needs endorsements, since it requires nodes to download
and install new software, and signal that they are ready to run the new version.
A consensus-update proposal has a \emph{safety lag} associated with it, which is
a time window that ensures sufficient time is provided to the parties to
download and deploy the candidate proposal. This safety lag determines the
duration of the endorsement period and \textbf{must be specified in number of
  epochs}. As a result, the end of the safety lag coincides with the end of an
epoch.

When a consensus-update proposal becomes a candidate, block producers can start
endorsing it. The stake associated with the keys endorsing the proposal is
tallied $2k$ blocks before the end of an epoch, where $k$ is maximum
\emph{number of blocks} the chain can roll back (see
Section~\ref{sec:on-cutoff-points-for-endorsements}). We consider the stake
\textbf{at the slot in which we tally}. We have two activation thresholds
depending on the epoch in which the tally takes place:
\begin{itemize}
\item If the next epoch does not coincides with the end of the safety lag, then
  this threshold is the \emph{adoption threshold} ($\tau_A$).
\item If the next epoch coincides with the end of the safety lag, then this
  threshold is set to $51\%$.
\end{itemize}

Once we sum up the stake endorsing a proposal there are three possibilities:
\begin{enumerate}
\item If the proposal has not gathered enough endorsing stake then:
  \begin{enumerate}
  \item If the safety lag expires on the next epoch then the proposal is
    canceled.
  \item If the safety lag does not expire on the next epoch then the proposal
    can continue to be endorsed on the next epoch. The endorsements are carried
    over.
  \end{enumerate}
\item If the proposal has gathered enough endorsing stake, then it is
  \emph{scheduled for activation} at the beginning of the next epoch.
\end{enumerate}

Consensus and parameters updates get activated \emph{at the beginning of an
  epoch}. The reason for this is that the ledger and consensus rules rely on
these to be stable during any given epoch.

The cancellation\footnote{Note that we classify expired proposals, i.e.
  proposals that did not get enough endorsements at the end of the safety lag,
  as canceled proposals.} or activation of a proposal causes its removal from
the activation queue, and the next implemented proposal in the queue, if any, to
become the new candidate and enter the endorsement period.

A proposal being endorsed can go back to the queue (see
Section~\ref{sec:the-queue}). In this case the endorsements for this proposal
are discarded.

\subsubsection{Parameters update}
\label{sec:parameters-update}

Parameters-update proposals determine new values for the (existing) protocol
parameters. The current version of the software can deal with this sort of
updates, and therefore there is no need for the nodes to download and install
software, and thus no need for an endorsement period. However, they can only be
activated at beginning of an epoch. Parameters-update must also wait in the
activation queue.

A protocol parameters update must have a corresponding approved SIP, which
justifies the need for changing the parameter values. The submitted
implementation (UP) specifies the new values and the new protocol version
associated to these parameters. Both SIP and UP describe the parameters change.
The SIP does this in an abstract way, e.g. ``change maximum block size to 1MB'',
whereas the UP describes this change in a concrete way, e.g.
\texttt{maxBlokcSize = 1024}.

Note how, unlike Byron, the activation of one parameter update does not
automatically eliminate all the other pending parameters updates. We rely on the
versioning scheme for resolving conflicts. For instance, assume the current
protocol version is \texttt{1.0.0}. If updates with versions \texttt{1.1.0} and
\texttt{1.2.0} both change a parameter \texttt{p}, activating \texttt{1.1.0}
does not necessarily causes \texttt{1.2.0} to be discarded. It all depends on
the version the \texttt{1.2.0} supersedes. If it supersedes \texttt{1.0.0}, then
it will be canceled since the current version is now \texttt{1.1.0} and versions
can only increase. However \texttt{1.2.0} could declare that it supersedes
version \texttt{1.1.0}, in which case it can be activated after this version
(for instance in the next epoch).

\subsubsection{Application updates}
\label{sec:software-only-updates}

Application updates are not related to protocol updates, and therefore the do
not enter the activation phase.
%
Upon approval of an application update the ledger simply records this event.
%
The dependencies between applications and protocol versions has to be managed
off chain, and it is the responsibility of developers (who implement the
updates) and experts. The update system can provide metadata for specifying the
dependencies and/or conflicts of an application with the rest of the ecosystem,
but the update system will not check this.

Just like the other kind of updates, application updates must have a
corresponding approved SIP and implementation.

\subsubsection{Explicit cancellation}
\label{sec:explicit-cancellation}

A proposal can be explicitly canceled. This allows the experts and community to
cancel an approved implementation after a (significant) problem with it was
discovered.

An explicit cancellation is submitted as a SIP that must justify the reason for
canceling the proposals. When submitted as an update proposal (UP), the
cancellation must specify the proposals that it will cancel if approved.

The explicit cancellation also goes through the ideation and approval phases,
however, the explicit cancellation \emph{immediately cancels} the proposals it
refers to once it gets approved by the expert pools in the approval phase.

Cancellation proposals have no next-versions associated with it. They only
specify the hash of the implementations they cancel.

\subsubsection{Cancellations}
\label{sec:cancellations}

There are several factors that can cause a proposal to be canceled:

\begin{itemize}
\item The proposal is explicitly canceled by a cancellation proposal that made
  it past the approval phase.
\item The proposal is overridden by another proposal with the same version that
  made it through the approval phase. If the experts approve such proposal, then
  this gives a strong indication that the older proposal should be discarded.
  Note that even though the system guarantees that only one proposal per-version
  will get approved in any given slot (see section \ref{sec:activation-phase} 
  on how we resolve the conflict of proposals of the same version approved in 
  the same slot), 
  there might be proposals approved with
  the same version as a proposal approved in a previous slot.
\item The proposal supersedes a version that can never follow. Protocol versions
  increase monotonically. So if an update depends on a version that is lower
  than the current version we know for sure that such version can never be seen
  on the chain, and therefore the proposal that depends on this version can be
  discarded.
\end{itemize}

If the candidate proposal already got the required endorsements (which means
that the cancellation arrived at or later than the slot at which the tally
occurred, this is $2k$ blocks before the end of an epoch), and is waiting to
be activated, then the cancellation update cannot stop it. In this case, a new
software update must be implemented and submitted that will correct the
identified problem.

If a candidate proposal is a parameters update, which means that does not need
endorsements, but the cancellation arrives \textbf{after} $2k$ blocks before
the end of the epoch, then it cannot be canceled either for the reasons given in
Section~\ref{sec:on-cutoff-points-for-endorsements}.

Nodes that upgraded to a canceled version can continue to operate normally
following the current version since the upgraded version \textbf{must} be able
to follow the current ledger and consensus rules.
%
Then it is up to the node operators to revert back to the previous version, or
continue using the software version that can also validate the protocol version
that got discarded. The ledger rules \textbf{shall ensure} that this discarded
protocol version is never applied.

\subsubsection{The queue}
\label{sec:the-queue}

Approved proposals enter an activation priority queue. The order of the queue is
determined by the proposal's versions. This queue \textbf{shall not} contain any
duplicated versions. If a proposal with the same version as a proposal from the
queue gets approved, then:

\begin{enumerate}
\item If the old proposal (the one in the queue\footnote{Note that only the
    proposal in front of the queue can receive endorsements.}) does not have
  enough endorsements by the time the new proposal is approved, then the old
  proposal is canceled.
\item If the old proposal has enough endorsements, then the new proposal is
  canceled.
\end{enumerate}


If a proposal with a lower version than the current candidate enters the queue
(i.e. the front of the queue changes), then we also have two situations:

\begin{enumerate}
\item If the candidate proposal does not have enough endorsements, then it is
  put back in the queue, and the new proposal becomes the new candidate. If this
  new candidate gets approved then the old proposal will be canceled (since it
  will supersede a version lower than the current version). However, since it is
  possible that the new proposal gets canceled (for instance due to lack of
  enough endorsements), we leave the old proposal in the queue.
\item If the candidate proposal has enough endorsements, then the new proposal
  gets discarded (since it will be superseding an older version that will never
  be adopted since versions increase monotonically).
\end{enumerate}

\subsubsection{On cut-off points for endorsements}
\label{sec:on-cutoff-points-for-endorsements}

The ledger layer must provide future information about a part of its state. In
particular, it should be able to tell up to $k$ blocks in the future if a
new protocol version will be activated. To this end, we require that the
\emph{last endorsement} required for meeting the adoption threshold \emph{is
  stable} $k$ blocks before the end of the epoch. This means that
endorsements for a given proposal for a given epoch are considered up to
$2k$ blocks before the end of that epoch. This is not to say that the
endorsements after this cut-off point will not be considered, just that they
will in the next epoch.

To see why this is required, consider a last endorsement (required to meet the
threshold) arriving between $2k$ blocks and $k$ blocks before the end of an
epoch, which will happen at block number $b_e$. If the ledger is asked between
blocks $b_e - k$ and $b_e$ whether a new version will be activated at $b_e$, the
answer depends on the stability of the last endorsement. If it is not stable,
then replying ``yes'' might be incorrect, since we can roll back and switch to a
fork where that endorsement never occurred. If we reply ``no'', then this might
also be incorrect if we continue in that fork.

\subsection{Open questions}
\label{sec:open-questions}

\begin{itemize}
\item Is the requirement that updates specify \textbf{exactly} the version it
  supersedes too restrictive? Do we foresee lifting this restriction any time
  soon? If so we might want to revisit the activation phase described here.
\item Is there a problem in supporting software updates for one code base only?
\item Do we see a problem with making the protocol permissive and allow any
  version to be endorsed? Invalid endorsements, i.e. endorsements that don't
  refer to an existing candidate version, will simply be discarded.
\end{itemize}

%%% Local Variables:
%%% mode: latex
%%% TeX-master: "decentralized-updates"
%%% End:


\section{Impact on performance}
\label{sec:impact-on-performance}

An important measurement of a blockchain system's performance is the number of
transaction bytes per-second ($\mathit{TBPS}$) it can sustain. Unlike the more
commonly used metric, transactions per-second, this number is not dependent on
the chosen size of a transaction (which would allow to manipulate it at will by
choosing different transactions sizes). Running an update mechanism on the
blockchain should not result in a substantial performance degradation.
Therefore, in this section we estimate the impact on performance that the
proposed update mechanism will have on a system's TBPS. Our estimations are
based on \emph{worst case scenarios}, which allow us to determine upper bounds
for the performance impact of the update mechanism.

Given a payload size in bytes ($\mathit{psize}$) that needs to be stored in a
blockchain, and the blockchain's throughput measured in $\mathit{TBPS}$, we have
that the processing time ($\mathit{ptime}$) for the given payload can be
calculated as:
$$ \mathit{ptime} = \frac{\mathit{psize}}{\mathit{TBPS}}$$

The processing time needs to be considered relative to the duration of the
update process ($\mathit{duration}_u$). If we require $\mathit{ptime}$ seconds
to process a payload of $\mathit{psize}$ bytes, and the update process lasts for
$\mathit{duration}_u$ seconds, then the percentage of the blockchain system's
time that will be occupied processing the update payload can be
calculated as:
$$\mathit{ptime}_{\mathit{rel}} = 100\frac{\mathit{ptime}}{\mathit{duration}_u}$$

Equivalently, we can consider the percentage of the system's $\mathit{TBPS}$
that will be used by the update payload:
$$\mathit{usage}_{\mathit{pct}} = 100\frac{\mathit{psize}}{\mathit{TBPS} ~ \mathit{duration}_u}$$

In the analysis that follows, we consider only $\mathit{usage}_{\mathit{pct}}$,
as the value of $\mathit{ptime}_{\mathit{rel}}$ is the same.

An update consists of several phases (ideation, implementation, approval, and
activation). In each phase, there are three types of messages being sent:
commits, reveals, and votes.

% Why do we consider only the voting period:
Before the voting phases, where votes can be cast by the participants, each
update requires only two messages spread across two stability windows, needed
for transactions to stabilize in the chain. These stability windows are quite
large, e.g. in Cardano the window lasts for $2k$ slots, $k=2160$, and a slot
lasts for $20$ seconds, which means that the stability window is 1 day. As a
result, only two messages need to be transmitted for the commit-reveal phase
over a large period of time, which means that a blockchain system can easily
handle this. This leave us with the voting phase as the sole source for
performance degradation that can be caused by the update mechanism.

% Why do we consider phases in isolation
In addition, note that we only need to consider the additional load introduced
by the update mechanism during a single phase. It is in the voting period of
each phase where the system should be able to handle the additional load, since
the update mechanism introduces very little load between voting phases.

We define the worst-case scenario for a voting period in terms of
\begin{itemize}
\item number of participants ($n_p$), e.g. voters (note that in the worst case
  scenario everybody will vote, regardless of their stake, which means that the
  stake distribution is irrelevant for this analysis)
\item number of update proposals being voted at the same time ($n_c$), during the same
  period (note that in the worst case scenario multiple update proposals will
  coincide in the start and end of the voting period, otherwise the system would
  have a larger time interval to distribute the load).
\item number of time a participant changes her vote ($n_c$), per-update proposal
\end{itemize}
Then, we can calculate the worst case scenario for the number of bytes that need
to be transmitted as part of the vote payload ($\mathit{psize}_v$) as:
$$\mathit{psize}_v = s_v \mathit{n_p} n_r n_c$$

The size in bytes for $s_v$ was obtained by calculating the size CBOR
encoding~\cite{RFC7049} of the vote payload of our prototype. This payload
includes:
\begin{itemize}
\item The hash of the voted SIP. We use 32 bytes hashes, so considering the 1
  byte CBOR tag this gives us a total 33 bytes.
\item The confidence (for, against, reject), which can be encoded in 1 byte
  (which also included the CBOR tag).
\item The key of the voter. We use 32 bytes keys, so this give us a total of 33
  bytes, when we consider the CBOR tag.
\item The vote signature. We consider 64 bytes signatures, which are accompanied
  by a 32 bytes key. This results in 64 + 32 + 1 bytes required for the
  signature.
\end{itemize}
So a vote requires in total 164 bytes.

Table~\ref{fig:tab:worst-case-analysis-voting-period} shows the results of the
worst case analysis for different parameter values, where $\mathit{duration}_v$
is the number of voting days, which was used to calculate the voting period
duration. For this analysis we use the (very conservative) $\mathit{TBPS}$
estimate that Cardano can achieve: $25k$\footnote{TODO: we need a reference for
  this}.

\begin{table}[htb]
  \centering
  % NOTE: DO NOT EDIT THE TABLE BELOW: It was generated by the benchmarking program. See `formal-spec/bench` folder.
  \begin{tabular}{| r | r | r | r | r |}
    \hline
    $n_p$ & $n_r$ & $n_c$ & $\mathit{duration_v}$ & $\mathit{usage_{pct}}$\\
    \hline
    1000 & 2 & 1 & 7 & 0.002\\
    1000 & 2 & 5 & 7 & 0.011\\
    1000 & 2 & 10 & 7 & 0.022\\
    10000 & 2 & 1 & 7 & 0.022\\
    10000 & 2 & 5 & 7 & 0.108\\
    10000 & 2 & 10 & 7 & 0.217\\
    100000 & 2 & 1 & 7 & 0.217\\
    100000 & 2 & 5 & 7 & 1.085\\
    100000 & 2 & 10 & 7 & 2.169\\
    1000000 & 2 & 1 & 7 & 2.169\\
    1000000 & 2 & 5 & 7 & 10.847\\
    1000000 & 2 & 10 & 7 & 21.693\\
    1000000 & 2 & 10 & 14 & 10.847\\
    1000000 & 2 & 10 & 30 & 5.062\\
    10000000 & 2 & 1 & 7 & 21.693\\
    10000000 & 2 & 5 & 7 & 108.466\\
    10000000 & 2 & 10 & 7 & 216.931\\
    \hline
  \end{tabular}
  \caption[Worst-case analysis for voting period]{Worst-case analysis TBPS for a voting period}
  \label{fig:tab:worst-case-analysis-voting-period}
\end{table}

Figure~\ref{fig:usage-vs-participants} shows the worst case usage as a function
of the number of participants, assuming 10 concurrent update proposals, a 7 day
voting period, and each participant changing her vote twice.

\begin{figure}[htp]
  \centering

  \begin{tikzpicture}
    \begin{axis}[
      title={Participants vs usage percentage},
      xlabel={Participants},
      xmin=100.0,
      xmax=1.0e7,
      xmode=log,
      xtick={10.0, 100.0, 1000.0, 10000.0, 100000.0, 1000000.0, 1.0e7},
      ylabel={Usage percentage},
      ymin=0.0,
      ymax=250.0,
      ymode=linear,
      ytick={5.0, 20.0, 50.0, 100.0, 150.0, 200.0, 250.0},
      legend pos=north west,
      ymajorgrids=true,
      grid style=dashed,
      ]
      \addplot[color=black] table {participants-vs-usage.dat};
    \end{axis}
  \end{tikzpicture}

  \caption{Worst case scenario analysis for system's usage with 10 concurrent update proposals}
  \label{fig:usage-vs-participants}
\end{figure}

We can see that the impact on the system's performance is negligible even when
we consider $100,000$ participants.
%
These results indicate that the update protocol will start degrading the system
performance \emph{only} past the 1,000,000 participants. Although this will
require that the worst case conditions being met: 10 SIP's being voted at the
same time over the period of 7 days, where each participant votes twice.
%
In such case, relying on \emph{voting pools} (or \emph{expert pools}) becomes of
crucial importance. In this way delegation of voting rights can help the update
system is to scale beyond this number of participants. Alternatively, by
increasing the duration of the vote period the impact on the system's
performance can be mitigated.

%%%%%%%%%%%%%%%%%%%%%%%%%%%%%%%%%%%%%%%%%%%%%%%%%%%%%%%%%%%%%%%%%%%%%%%%%%%%%%%%
%% TODO: we need the revisit the section below considering the insight of
%% the section above.
%%%%%%%%%%%%%%%%%%%%%%%%%%%%%%%%%%%%%%%%%%%%%%%%%%%%%%%%%%%%%%%%%%%%%%%%%%%%%%%%

\section{Measurements specification} \label{sec:measurements}

In this section we want to describe an experimental evaluation of our proposed
update mechanism. This experimental evaluation will help to verify to what
degree our proposal fulfills non-functional requirements, such as the ones
described in section \ref{sec:non-func-reqs}.

\subsection{What to measure} \label{sec:what-to-measure} Our experimental
evaluation will mainly focus on the following metrics:

\paragraph{Transaction throughput}
\emph{High-level goal:} We want to evaluate the impact of the update protocol to
the transaction throughput of the blockchain system, i.e., to the number of
transactions that manage to get into a block in the unit of time.

Let be $N_{Tx}$ the average number of transactions that fit in a block. It is
defined as
%
$$N_{Tx} = \frac{\text{Block Size}}{\text{Avg Transaction Size}}$$
%
If a new block is issued every $T_B$ units of time, then we define the
\emph{transaction throughput} $Tx_{th}$ as the ratio
%
$$Tx_{th} = \frac{N_{Tx}}{T_B}$$
%
and we usually measure it in \emph{transactions per sec (tps)}. We want to
evaluate the impact of the number of users $N_u$ that actively participate in
the update mechanism to this metric.

\paragraph{Blockchain size}
\emph{High-level goal:} We want to evaluate the impact of storing update
transactions (i.e., transactions with an update payload) within a block, to the
number of common transactions that can be stored in the unit of storage.

Lets consider a blockchain system running a consensus protocol. At time point
$T_{start}$ we omit $k$ blocks from the end of the chain ($k$ is the security
parameter of the protocol) and mark the slot of the last block in the remaining
chain as $S_{start}$. We let the consensus protocol run for a fixed time window
of $T_w$ units of time until $T_{end}$ ($T_w = T{end} - T{start}$). Similarly,
we omit $k$ blocks from the end of the chain and mark the slot of the first
block in the remaining chain as $S_{end}$. We define the size of the chain
between slot $S_{start}$ and slot $S_{end}$ (included), as the \emph{blockchain
  size of time window} $T_w$ and call it $BSize_{T_w}$. Let $N_{T_w}$ be the
number of transactions stored in the chain of size $BSize_{T_w}$. We define the
\emph{number of transactions per unit of storage}, denoted $Tx_{s}$, as:
%
$$
Tx_{s} = \frac{\mathit{N_{T_w}}}{\mathit{BSize}}
$$
%
We want to evaluate the impact of the number of users $N_u$ that actively
participate in the update mechanism to this metric.

\paragraph{Update time to activation}
\emph{High-level goal:} we want to measure the total time to complete a software
update (i.e., the end-to-end elapsed time from submission-of-proposal to
activation) and evaluate how is this time impacted by the the number of active
users participating in the update protocol.

A software update (SU) starts its life with the submission into the blockchain
of a \emph{system improvement proposal} (SIP) and ends upon the activation of
the SU. We define the elapsed time that takes for a software update to activate
from start to finish, excluding the human delays (e.g., the time it takes for
the implementation of the software update) as the \emph{update time to
  activation} and note $T_{act}$. We want to evaluate the impact of the number
of users $N_u$ that actively participate in the update mechanism to this metric.

\paragraph{Processing time}
\emph{High-level goal:} we want to measure the computation load of the update
protocol and identify the most computation-heavy phases of the update protocol.

To this end, we will measure the processing time spent by the node, in each
\emph{distinct phase} of the decentralized software update \emph{lifecycle}, as
the number of active users $N_u$ participating in the update protocol increases.

\subsection{How to measure}
We consider two different approaches for conducting the measurements, depending
on whether we integrate with the Cardano blockchain and the underlying consensus
protocol. In the \emph{single node} alternative, we have a single node
implementation that does not need to run the consensus protocol; it only runs
the software updates protocol over a generated set of update events. In the
\emph{networked node} alternative, multiple nodes run the consensus protocol, as
well as the software updates protocol and communicate with each other.

In the single node case, the blockchain is generated by the generator process
and not by the node itself. The node reads each generated block and runs the
update protocol, since the block contains also transactions with update payload
apart from common transactions. So the update events stored in each block,
trigger a corresponding action from the part of the node that has to do with the
update protocol.

In the networked node case, the generator generates only transactions and not
blocks, which then transmits to the network to the running nodes. The nodes
listen to the network for these transactions and based on the rules of the
consensus protocol create blocks, which are then transmitted again to the
network for the other nodes to receive. Every node builds a local blockchain
based on the consensus protocol. We assume that apart from common transactions,
also transactions with update payload are generated, which trigger a node to run
the update protocol (as in the single-node case), but in this case this is done
along with the execution of the consensus protocol.

\subsubsection{Single node simulation}
\paragraph{Justification for the single node simulation}
In a realistic networked implementation of the update protocol, each participant
node will maintain a local blockchain consisting of common, as well as, update
transactions. We argue that for the specific measurements at hand (see section
\ref{sec:what-to-measure}) all the entailed information that we need, in order
to measure them correctly, has been incorporated in the end-product, which is
the local blockchain maintained by each node. Indeed, the number of common
transactions that managed to get through in the unit time of time, as well as
the number of common transactions that managed to get stored in a block, in the
presence of transactions with update payload, has been fully captured in the
produced blockchain. Therefore, we really do not need to actually run the
consensus layer in order to measure our metrics, but we need the result of the
consensus layer and that of the execution of the update protocol. As long as the
produced blockchain realistically represents such as an execution, the single
node measurements should be equivalent to the networked ones. The only
difference between the two approaches is that in the networked case, in some
cases (depends on the actual network setup) it is easier to produce a more
realistic blockchain end-product.

\paragraph{Set up}
We assume a single generator process. This generator process will simulate $N_u$
users running the software update protocol thus generating update events (i.e.,
update transactions) and $M_u$ users running the consensus protocol generating
common transactions.

We assume that a single user generates update transactions with a specific
weight compared to common transactions (e.g., 1 update transaction for every
1000 common transactions). As we increase the number of users actively
participating in the update protocol the weight that corresponds to the
generation of update events also increases. So the ratio of update transactions
to common transaction increases e.g., $r_u = 1/1000, 2/1000, ...$

The generator will produce a fixed length blockchain consisting of blocks that
store either common transactions, or update transactions. This blockchain will
trigger the node to run for a specific time window, either processing common
transactions, or running the update protocol. As the above ratio increases, we
expect to see the trace including more and more update transactions and less
common transactions. This is expected also, to impact at some point the running
time of the node, since the node will have to do more work due to the software
updates, but only at specific periods of the update protocol (e.g., the tally
phase) where the computation requirements are more intense.

It is important in the produced blockchain to simulate a realistic analogy of
stored transactions per block. To this end, in our size calculations we use the
actual Cardano \emph{maximum block size} $B_{max}$. In order to achieve a
realistic transactions per block analogy, we consider the
$TxPerBlock = \frac{Maximum Block Size}{Average Transaction Size}$, similarly
based on the Cardano benchmarks. So for example, if $TxPerBlock = 800$ and
$B_{max} = 1MB$, then we assume an abstract storage cost equal to $800$ units
for a block and an abstract storage cost of $1$ unit for a common transaction.
Then, for a transaction with an update payload, we consider an abstract storage
cost equal to $\frac{Actual Size of Update Payload}{B_{max}}$. In this way, our
abstract storage sizes simulate a realistic block size and more importantly, a
realistic analogy of common transactions and update transactions fitting in a
block.

\paragraph{Transaction throughput}
The number of common transactions included in a generated trace (i.e., a
blockchain consisting of common transactions and update transactions) of a
specific fixed length and corresponding to a specific ratio $r_u$, divided by
the total processing time of the node, will be the logical equivalent to the
transaction throughput in this setup. To compute the number of common
transactions, we only need to scan the produced blockchain. Similarly, to
compute the total processing time we multiply the number of blocks produced, by
the average time to produce a block (based on a Cardano benchmark measurement).

As more update transactions are added to the blockchain, the number of common
transactions stored in this fixed length blockchain will
decrease, %and also the processing time of the node will increase%
so we expect that the transaction throughput will decrease. We want to evaluate
experimentally the rate of this decrease as the $r_u$ ratio scales up.

\paragraph{Blockchain size}
The number of common transactions included in a generated trace (i.e., a
blockchain consisting of common transactions and update transactions) of a
specific fixed length and corresponding to a specific ratio $r_u$, divided by
the size of this blockchain, will be the measured Blockchain Size metric in this
setup. The number of common transactions is calculated as in the previous
metric. In order to compute the size of the produced blockchain, we take the
number of produced blocks and multiply them by the maximum size of a block
(based on the Cardano implementation).

As more update transactions are added to the blockchain the number of common
transactions per unit of storage is expected to decrease. We want to evaluate
experimentally the rate of this decrease as the $r_u$ ratio scales up.

\paragraph{Update time to activation}
We generate a fixed trace (i.e., a blockchain consisting of common transactions
and update transactions) of a specific fixed length and corresponding to a
specific ratio $r_u$. We scan this trace and for each SU encountered we
calculate the total elapsed time. To do this, we will use a \emph{representative
  transaction processing time} (e.g., time to block a transaction
$\approx 20 secs$), which will come from the Cardano benchmarks. For each update
transaction of a specific software update encountered, we add this
representative transaction processing time and thus calculate a total elapsed
time for the software update to be activated. Of course, we also need to take
into account other time periods like the stability windows, the voting periods
etc. So we will also need a \emph{representative time for a slot to be
  generated}, since all these periods correspond to specific number of slots.
Once we calculate the elapsed time for each software update, then we calculate
the 80th-percentile as the representative \emph{update time to activation}. As
more update transactions are added to the blockchain, this metric is expected to
increase. We want to evaluate experimentally the rate of this increase as the
$r_u$ ratio scales up.

\paragraph{Processing time}
We generate a fixed trace (i.e., a blockchain consisting of common transactions
and update transactions) of a specific fixed length and corresponding to a
specific ratio $r_u$. This trace will trigger the execution of the update
protocol by the node. For each distinct phase of the software update lifecycle,
we measure the actual processing time of the node. We want to evaluate the
processing time per distinct phase in order to identify the most
computation-heavy phases of the update protocol. As more update transactions are
added to the blockchain, the processing time is expected to increase, as more
work is done by the node. We want to evaluate experimentally the rate of this
increase as the $r_u$ ratio scales up.

\subsubsection{Networked node simulation}
\paragraph{Set up}
\paragraph{Transaction throughput}
\paragraph{Blockchain size}
\paragraph{Update time to activation}


\section{Validation criteria}
\label{sec:validation-criteria}

This section describes the tests performed on the update protocol implementation
to validate its correctness.

The update mechanism is modeled using data-automata and implemented as a set of
structured-operational semantic rules~\cite{7436dacbd7664737925923f0657fdc07,
  quickcheck-state-machine}. The conformance of the implementation w.r.t. its
specification is tested by means of property-based
tests~\cite{10.1145/1988042.1988046}.

In this section we introduce the data-automata formalism, show how conformance
of the implementation w.r.t. automata models is carried out, and present the
data-automata models that capture the properties described in
Section~\ref{sec:cand-properties}.

\subsection{The data automata formalism}
\label{sec:the-data-automata-formalism}

TODO.

\subsection{Conformance tests}
\label{sec:conformance-tests}

TODO.

\subsection{Data-automata models for the update protocol}
\label{sec:automata-models-update-protocol}


\begin{itemize}
\item We assume a correspondence between natural numbers and SIP's.
\item Models are quantified over SIP's.
\end{itemize}

TODO: show full model (quantified over the maximum number of SIP's).

\subsubsection{Ideation phase}
\label{sec:validation-ideation-phase}

\paragraph{Property~\ref{prop:unique-sip}} We have validated that the
implementation conforms to the commit-reveal model shown in Equation~??? and
Figure~???, which is part of the full model of Equation~???. The model in
Equation~??? can perform a $\mathsf{submit}_i$ action only once. Here it is
clear that, since this model synchronizes on the $\mathsf{submit}_i$ action,
which means that no other automata can submit the same SIP without synchronizing
with this automata, therefore in the full model a submission for an SIP can
occur at most once (remember that since we test conformance of the rules w.r.t.
invalid automata traces as well, this means that duplicated SIP's should also be
rejected by a system that conforms to the full model of Equation~???).

TODO: write the equation corresponding to the commit reveal automaton
synchronizing on submit and reveal action.

TODO: generate commit-reveal figure from Haskell implementation.

\paragraph{Property~\ref{prop:reveal-valid}} This property is also captured by
the commit-reveal automata discussed in the previous section (Equation~??? and
Figure~???).

\paragraph{Property~\ref{prop:active-sip-valid}} This is captured by the
\texttt{activation} model described in Equation~??? and Figure~??? As it can be
observed in this model, SIP's are only active once the $\mathsf{reveal}_i$
signal is stable, and time can pass up to the point in which the end of the
voting period ends. After that, the SIP is not active anymore.

TODO: generate the activation model and add an equation with the corresponding
synchronizing actions.

\paragraph{Property~\ref{prop:sip-votes-valid}} This requirement also is
validated by showing conformance to the \texttt{activation} model.

\paragraph{Property~\ref{prop:sip-tally-valid}} This requirement is validated by
the parallel composition and synchronization of the \texttt{activation} and
\texttt{voteTally} models. Here the $\mathsf{tally}_i$ action can only occur
after the end of the vote period is stable, and stake updates are not possible
while tallying, which means that the stake considered when tallying is the state
at the slot when the end of the voting period stabilized. Note that time could
pass while tallying (we do not restrict this), we only care about the snapshot
of the stake we use in tallying.

\paragraph{Property~\ref{prop:sip-tally-result}} This requirement is validated
by the \texttt{voteTally} model.

\paragraph{Property~\ref{prop:sip-revote}} TODO: we need to model revoting.


%%% Local Variables:
%%% mode: latex
%%% TeX-master: "decentralized-updates"
%%% End:


\addcontentsline{toc}{section}{References}
\bibliographystyle{acm}
\bibliography{references}

\end{document}
