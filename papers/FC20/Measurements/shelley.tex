\section{Shelley Prerequisites}\label{shelley}

\subsection{Assumptions}
\paragraph{Number of active users $u$.}
We assume that $u$ represents the number of users \emph{actively} participating in the software updates protocol. An active user is a user that is on-line and generates protocol events, such as: 
\begin{itemize}
\item submits SIPs
\item votes for SIPs
\item submits UPs
\item votes for UPs
\item signals upgrade readiness
\end{itemize}

\subsection{Event Submission Time $est(u)$}
All events (proposal submission, votes etc.) generated by the software update protocol are essentially embedded within common transactions and transmitted to the network before being committed into the blockchain. Therefore, for all of our time measurements, we need to be able to calculate the time for the submission of a transaction into the blockchain network. 

In particular, we define a function $\textbf{est(u)}$, which returns the \emph{Event Submission Time}, when a specific number of $u$ users participate in the protocol (i.e., \emph{active} users). The $est$ comprises the following steps in the life of a transaction:
\begin{itemize}
\item The transaction $T_x$ that embeds a software update protocol event, is created by the user and transmitted to the network.
\item $T_x$ is selected from the mempool and included into a block by a slot leader (i.e., miner). This block is then transmitted to the network.
\item Stabilization period. This is the required time period, in order for the new block to be stably committed into the blockchain. This period is a function of the security parameter $k$. We assume the existence of a function $stable(k)$ that returns this period (see next). 
\end{itemize}

The $est(u)$ function might \emph{not} be a pure function. Meaning that for a specific number of users might not return a fixed time. The result however must correspond to a network of $u$ active users.

\subsection{Signal Submission Time $sst(u)$}
We define a function $\textbf{sst(u)}$, which returns the \emph{Signal Submission Time}, when a specific number of $u$ users participate in the protocol (i.e., \emph{active} users). The $sst$ comprises the following steps:
\begin{itemize}
\item A slot leader creates a new block and marks it with the new version (after the upgrade) and transmits it to the network.
\item Stabilization period. This is the required time period, in order for the new block to be stably committed into the blockchain. This period is a function of the security parameter $k$. We assume the existence of a function $stable(k)$ that returns this period (see next). 
\end{itemize}

The $sst(u)$ function might \emph{not} be a pure function. Meaning that for a specific number of users might not return a fixed time. The result however must correspond to a network of $u$ active users.


\subsection{Stabilization Period $stable(k)$}
Stabilization period. This is the required time period, in order for a new block to be stably committed into the blockchain. This period is a function of the security parameter $k$. We assume the existence of a function $\textbf{stable(k)}$ that returns this period.