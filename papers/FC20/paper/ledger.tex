
\subsection{The blockchain abstraction}\label{se:bcabstraction}

Following~\cite{treasury}, we abstract the following concepts.
\begin{itemize}
	\item Coin. We assume the underlying blockchain platform has the notion of
Coins or its equivalent. Each coin can be spent only once, and all the value of
coin must be consumed. Each coin consists of the following
4 attributes:
	\begin{itemize}
		\item[-]Coin ID: it is an implicit attribute, and every coin has a unique ID that
can be used to identify the coin.
		\item[-] Value: It contains the value of the coin.
		\item[-]Cond: It contains the conditions under which the coin can be spent.
		\item[-] Payload: It is used to store any non-transactional data.
	\end{itemize}
	\item Address: conventionally, an
		address is merely a public key, $\pk$, or hash of a public key, $h(\pk)$. To create
coins associated with the address, the spending condition of the coin should be
defined as a valid signature under the corresponding public key $\pk$ of the address.
In this work, we define an address as a generic representation of some spending
condition. Using the recipient address, a sender is able to create a new coin
whose spending condition is the one that the recipient intended; therefore, the
recipient may spend the coin later.
	\item Transaction: Each transaction takes one or more (unspent) coins, denoted
		as $\{\inp\}_{i\in [n]}$, as input, and it outputs one or more (new) coins, denoted as
 $\{\out\}_{j\in [m]}$. Except special transactions, the following condition holds:

		$$\sum_{i=1}^n{\inp_i.\Value} \geq \sum_{j=1}^m {\out_j.\Value}$$
		and the difference is interpreted as transaction fee.
		The transaction has a Verification data field that contains the necessary verification data to satisfy all the spending conditions of the input coins $\{\inp\}_{i\inp[n]}$. In addition, each transaction also has a Payload field that can be used to store any non-transactional data. We denote a transaction as $\tx(A; B; C)$, where $A$ is the set
of input coins, $B$ is the set of output coins, and $C$ is the Payload field. Note that
the verification data is not explicitly described for simplicity.
\end{itemize}

